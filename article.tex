%  article.tex (Version 2.81, released 24 September 2003)
%  Article to demonstrate format for SPIE Proceedings
%  Special instructions are included in this file after the
%  symbol %>>>>
%  Numerous commands are commented out, but included to show how
%  to effect various options, e.g., to print page numbers, etc.
%  This LaTeX source file is composed for LaTeX2e,
%  not the older LaTeX version 2.09, as previous versions were.

%  The following commands have been added in the SPIE class
%  file (spie.cls) and will not be understood in other classes:
%  \supit{}, \authorinfo{}, \skiplinehalf, \keywords{}
%  The bibliography style file is called spiebib.bst,
%  which replaces the standard style unstr.bst.

%% \documentclass[]{spie}  %>>> use for US letter paper
\documentclass[a4paper]{spie}  %>>> use this instead for A4 paper
\addtolength{\voffset}{9mm}   %>>> moves text field down

%  The following command loads a graphics package to include images
%  in the document. It may be necessary to specify a DVI driver option,
%  e.g., [dvips], but that may be inappropriate for some LaTeX
%  installations.
\usepackage[]{graphicx}

\title{On Implementing eBook-related Web Standards \\ on Smart Devices}

%>>>> The author is responsible for formatting the
%  author list and their institutions.  Use  \skiplinehalf
%  to separate author list from addresses and between each address.
%  The correspondence between each author and his/her address
%  can be indicated with a superscript in italics,
%  which is easily obtained with \supit{}.

\author{Minhyung Ko and Soo-Hyun Choi\\
% \skiplinehalf
Samsung Electronics, Co., Ltd. \\
\{\textsf{minhyung.ko, sh9.choi\}@samsung.com}\\
}

%>>>> Further information about the authors, other than their
%  institution and addresses, should be included as a footnote,
%  which is facilitated by the \authorinfo{} command.

% \authorinfo{Further author information: (Send correspondence to A.A.A.)\\A.A.A.: E-mail: aaa@tbk2.edu, Telephone: 1 505 123 1234\\  B.B.A.: E-mail: bba@cmp.com, Telephone: +33 (0)1 98 76 54 32}
%%>>>> when using amstex, you need to use @@ instead of @


%%%%%%%%%%%%%%%%%%%%%%%%%%%%%%%%%%%%%%%%%%%%%%%%%%%%%%%%%%%%%
%>>>> uncomment following for page numbers
\pagestyle{plain}
%>>>> uncomment following to start page numbering at 301
% \setcounter{page}{1}

\begin{document}
\maketitle

%%%%%%%%%%%%%%%%%%%%%%%%%%%%%%%%%%%%%%%%%%%%%%%%%%%%%%%%%%%%%
\begin{abstract}
This document shows interest in participating the W3C workshop on ``eBooks: 
Great Expectations for Web Standards''.
\end{abstract}

%>>>> Include a list of keywords after the abstract

% \keywords{eBook, CSS3 Regions and Exclusions, MathML}

%%%%%%%%%%%%%%%%%%%%%%%%%%%%%%%%%%%%%%%%%%%%%%%%%%%%%%%%%%%%%
\section{INTRODUCTION}
\label{sect:intro}  % \label{} allows reference to this section
We are currently working to develop best practices and guidelines for eBook 
reader on smart devices. We hope to contribute to the W3C standardization 
activities for eBook and the Open Web Platform in this respect. We are also 
working to create devices for eBook that comply with the various W3C standards, 
for example CSS3 Regions and Exclusions. We are particularly interested in 
discussing about layout definition and control, accessibility, widgets 
definitions, and standardization and conformance.


%%%%%%%%%%%%%%%%%%%%%%%%%%%%%%%%%%%%%%%%%%%%%%%%%%%%%%%%%%%%%
\section{Point of View} \label{sect:view}

\begin{itemize}

    \item \textsf{W3C Standards like HTML5, SVG, MathML, Web APIs, MetaData, 
        etc.}  \\
        To this date, various requirements for eBook have already been addressed 
        by the HTML5 and CSS3 W3C standards. We are actively contributing to the 
W3C HTML5 specification, and we would like to contribute more in the eBook 
related standards as well going forward.

    \item \textsf{Layout definition and control} \\
        We would like to seamlessy support rendering of the eBooks on devices 
        with different screen resolutions and physical sizes. We would like to 
        initiate a discussion on how to achieve the same.

\end{itemize}


%%%%%%%%%%%%%%%%%%%%%%%%%%%%%%%%%%%%%%%%%%%%%%%%%%%%
\section{Proposed Participants} \label{sect:ppl}

\begin{itemize}
    \item Min-Hyung Ko \\
        {\small S/W Engineer \\
        Next Generation S/W R\&D Group, \\
        Mobile Communications Division, \\
        Samsung Electronics Co., Ltd.}\\
        \textsf{minhyung.ko@samsung.com}

    \item Soo-Hyun Choi, PhD \\
        {\small Senior S/W Engineer \\
        Next Generation S/W R\&D Group, \\
        Mobile Communications Division, \\
        Samsung Electronics Co., Ltd.}\\
        \textsf{sh9.choi@samsung.com}
\end{itemize}

%%%%%%%%%%%%%%%%%%%%%%%%%%%%%%%%%%%%%%%%%%%%%%%%%%%%%%%%%%%%%
% \acknowledgments     %>>>> equivalent to \section*{ACKNOWLEDGMENTS}

% This unnumbered section is used to identify those who have aided the authors in 
% understanding or accomplishing the work presented and to acknowledge sources of 
% funding.

%%%%%%%%%%%%%%%%%%%%%%%%%%%%%%%%%%%%%%%%%%%%%%%%%%%%%%%%%%%%%
%%%%% References %%%%%

% \bibliography{report}   %>>>> bibliography data in report.bib
% \bibliographystyle{spiebib}   %>>>> makes bibtex use spiebib.bst

\end{document}
